% ##################################################
% Unterstuetzung fuer die englishe Sprache
% ##################################################
\usepackage[english]{babel}
\usepackage{graphicx}
\usepackage{wrapfig}

% ##################################################
% Dokumentvariablen
% ##################################################

% Persoenliche Daten
\newcommand{\docNachname}{Gilli}
\newcommand{\docVorname}{Stephan}
\newcommand{\docStrasse}{Lerchenauerstraße 219b}
\newcommand{\docOrt}{München}
\newcommand{\docPlz}{80395}
\newcommand{\docEmail}{stephan.gilli@hs-furtwangen.de}
\newcommand{\docMatrikelnummer}{266674}

% Dokumentdaten
\newcommand{\docTitle}{The digital twin as a central control element in value networks}
\newcommand{\docArtDerArbeit}{Masterthesis}
% Software-Produktmanagement Bachelor, Advanced Computer Scinece Master
\newcommand{\docStudiengang}{Sales \& Service Engineering}
\newcommand{\docAbgabedatum}{28.02.2022}
\newcommand{\docErsterReferent}{Prof. Dr. Harald Kopp}
%\newcommand{\docZweiterReferent}{-} % Wenn es nur einen Betreuer gibt
\newcommand{\docZweiterReferent}{Prof. Dr.  Christoph Uhrhan}

% ##################################################
% Allgemeine Pakete
% ##################################################

% Abbildungen einbinden
\usepackage{graphicx}

% Zusaetsliche Sonderzeichen
\usepackage{dingbat}

% Farben
\usepackage{color}
\usepackage[usenames,dvipsnames,svgnames,table]{xcolor}

% Maskierung von URLs und Dateipfaden
\usepackage[hyphens]{url}

% Deutsche Anfuehrungszeichen
\usepackage[babel, german=quotes]{csquotes}

% Pakte zur Index-Erstellung (Schlagwortverzeichnis)
\usepackage{index}
\makeindex


% ##################################################
% Seitenformatierung
% ##################################################
\usepackage[
	portrait,
	inner=2.5cm,
	outer=2.5cm,
	top=3cm,
	bottom=2cm,
	]{geometry}

% ##################################################
% Kopf- und Fusszeile
% ##################################################

\usepackage{fancyhdr}

\pagestyle{fancy}
\fancyhf{}
\fancyhead[EL,OR]{\sffamily\thepage}
\fancyhead[ER,OL]{\sffamily\nouppercase{\leftmark}}

\fancypagestyle{headings}{}

\fancypagestyle{plain}{}

\fancypagestyle{empty}{
  \fancyhf{}
  \renewcommand{\headrulewidth}{0pt}
}

%Speichert \chaptermark in \oldchaptermark damit 
% es für die Anhänge zurückgesetzt werden kann
\let\oldchaptermark\chaptermark

%Kein "Kapitel # NAME" in der Kopfzeile
\renewcommand{\chaptermark}[1]{
	\markboth{#1}{}
   	\markboth{\thechapter.\ #1}{}
}

% ##################################################
% Schriften
% ##################################################

% Stdandardschrift festlegen
\renewcommand{\familydefault}{\sfdefault}

% Standard Zeilenabstand: 1,5 zeilig
\usepackage{setspace}
\onehalfspacing 

% Schriftgroessen festlegen
\addtokomafont{chapter}{\sffamily\large\bfseries} 
\addtokomafont{section}{\sffamily\normalsize\bfseries} 
\addtokomafont{subsection}{\sffamily\normalsize\mdseries} 
\addtokomafont{caption}{\sffamily\normalsize\mdseries} 

%Einrücken von Absätzen deaktivieren
\setlength{\parindent}{0pt}

%Zeilenabstand bei abstätzen
\usepackage{parskip}

% ##################################################
% Inhaltsverzeichnis / Allgemeine Verzeichniseinstellungen
% ##################################################

\usepackage{tocloft}

% Punkte auch bei Kapiteln
\renewcommand{\cftchapdotsep}{3}
\renewcommand{\cftdotsep}{3}

% Schriftart und -groesse im Inhaltsverzeichnis anpassen
\renewcommand{\cftchapfont}{\sffamily\normalsize}
\renewcommand{\cftsecfont}{\sffamily\normalsize}
\renewcommand{\cftsubsecfont}{\sffamily\normalsize}
\renewcommand{\cftchappagefont}{\sffamily\normalsize}
\renewcommand{\cftsecpagefont}{\sffamily\normalsize}
\renewcommand{\cftsubsecpagefont}{\sffamily\normalsize}

%Zeilenabstand in den Verzeichnissen einstellen
\setlength{\cftparskip}{.5\baselineskip}
\setlength{\cftbeforechapskip}{.1\baselineskip}

%Einrücken von Absätzen deaktivieren
%\setlength{\parindent}{0pt}

%Zeilenabstand bei abstätzen
\usepackage{parskip}

% ##################################################
% Abbildungsverzeichnis und Abbildungen
% ##################################################

\usepackage{caption}

\usepackage{wrapfig}

% Nummerierung von Abbildungen
\renewcommand{\thefigure}{\arabic{figure}}
\usepackage{chngcntr}
\counterwithout{figure}{chapter}

% Abbildungsverzeichnis anpassen
\renewcommand{\cftfigpresnum}{Figure }
\renewcommand{\cftfigaftersnum}{:}

% Breite des Nummerierungsbereiches [Abbildung 1:]
\newlength{\figureLength}
\settowidth{\figureLength}{\bfseries\cftfigpresnum\cftfigaftersnum}
\addtolength{\figureLength}{2mm} %extra offset
\setlength{\cftfignumwidth}{\figureLength}
\setlength{\cftfigindent}{0cm}

% Schriftart anpassen
\renewcommand\cftfigfont{\sffamily}
\renewcommand\cftfigpagefont{\sffamily}

%standardpfad anpassen
\graphicspath{ {../src/content/pictures/} }

% ##################################################
% Tabellenverzeichnis und Tabellen
% ##################################################

% Nummerierung von Tabellen
\renewcommand{\thetable}{\arabic{table}}
\counterwithout{table}{chapter}

% Tabellenverzeichnis anpassen
\renewcommand{\cfttabpresnum}{Table }
\renewcommand{\cfttabaftersnum}{:}

% Breite des Nummerierungsbereiches [Abbildung 1:]
\newlength{\tableLength}
\settowidth{\tableLength}{\bfseries\cfttabpresnum\cfttabaftersnum}
\addtolength{\tableLength}{3mm} %extra offset
\setlength{\cfttabnumwidth}{\tableLength}
\setlength{\cfttabindent}{0cm}

%Schriftart anpassen
\renewcommand\cfttabfont{\sffamily}
\renewcommand\cfttabpagefont{\sffamily}

% Unterdrueckung von vertikalen Linien
\usepackage{booktabs}

%Multi row für spezifische zellen
\usepackage{multirow}

%Additional table package
\usepackage{tabu}
  	
% ##################################################
% Literaturverzeichnis
% ##################################################

\usepackage{bibgerm}

% ##################################################
% Abkuerzungsverzeichnis
% ##################################################

\usepackage[printonlyused]{acronym}

% ##################################################
% PDF / Dokumenteninternelinks
% ##################################################

\usepackage[
	colorlinks=false,
   	linkcolor=black,
   	citecolor=black,
  	filecolor=black,
	urlcolor=black,
    bookmarks=true,
    bookmarksopen=true,
    bookmarksopenlevel=3,
    bookmarksnumbered,
    plainpages=false,
    pdfpagelabels=true,
    hyperfootnotes,
    pdftitle ={\docTitle},
    pdfauthor={\docVorname~\docNachname},
    pdfcreator={\docVorname~\docNachname}]{hyperref}

% ####################################################
% Command für einfache Quellenangabe bei Bilder, etc.
% ####################################################

\newcommand{\source}[1]{\caption*{Source: {#1}} }

% ###################################################
% Util Packages
% ###################################################
\usepackage{subfigure}

