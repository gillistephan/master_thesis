\chapter*{Abstract\markboth{Abstract}{}}
\addcontentsline{toc}{chapter}{Abstract}

In recent years, the interest in Industry 4.0 has increased rapidly, both in science as well as in industry. Industry 4.0 describes the vision of a global network in which machines, robots, warehousing systems and humans can autonomously interact with each other in order to control physical resources and processes. This is made possible in particular by the Industrial Internet of Things, in which a large number of devices such as RFID-readers, wireless sensors or cameras collect data on specific environmental conditions of assets. The autonomous interaction of participants and components in the value chain transforms production facilities into smart factories. A central technology in the transformation to smart factories is the digital twin. The digital twin allows the description of physical objects in the virtual world.  In the context of Industry 4.0, the digital twin is represented by the Asset Administration Shell, which is defined by the reference model RAMI 4.0. The Asset Administration Shell plays a central role in the implementation of RAMI 4.0, as it allows a technology-independent description of an asset in the virtual word according to defined standards. This thesis explores the role of the digital twin in the context of Industry 4.0 by highlighting the key features of the Asset Administration Shell and relating them to RAMI 4.0. Specifically, the role of the Asset Administration Shell in value networks is examined. Based on the key features and the role of the Asset Administration Shell in value networks, a technology-independent concept and guidance for the implementation is presented.

