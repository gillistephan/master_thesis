\chapter{Introduction} \label{chap:introduction}
This chapter first gives an overview of the motivation and background of the thesis. Subsequently, the research questions are derived and expected results are considered in more detail. Finally, a summary of the structure and chapters is given.

\section{Motivation}
In recent years, the interest in \ac{I4.0} has increased rapidly, both in science as well as in industry. \ac{I4.0} describes the vision of a global network in which machines, robots, warehousing systems and humans can autonomously interact with each other in order to control physical resources and processes. This is made possible in particular by the \ac{IIOT}, in which a large number of devices such as RFID-readers, wireless sensors or cameras collect data on specific environmental conditions of assets. The autonomous interaction of participants and components in the value chain transforms production facilities into smart factories \cite[p. 20]{Acatech2013Recommendations4.0}. Smart factories enable end-to-end engineering by connecting the digital and physical world both, horizontally and vertically along the value chain, but in particular across company boundaries \cite[p. 859]{Uslander2015ReferenceApproach}. This gives rise to new forms of collaboration such as value networks. The components and systems of smart factories should be able to make autonomous decisions based on provided information, execute functions regarding their capabilities, as well as autonomously influence production, order or logistic processes. For this reason, autonomously acting components and systems based on \ac{IIOT} technology are seen as the key enabler for \ac{I4.0}. They will have far-reaching effects on business and production processes. One term that is often mentioned in this context is that of a \ac{CPS}. A \ac{CPS} describes the networking of the physical world of machines, plants and devices with the virtual world \cite{Lee2008CyberBerkeley}. The terms smart factory and \ac{CPS} are therefore used interchangeably in this thesis.

Smart factories will open up new use cases for companies of various industries and sectors. In particular the following should be mentioned:
\begin{itemize}
    \item[] \textbf{Plug and produce} Plug and produce describes the ability of a production resource to configure itself, based on its environment conditions and integrate itself correctly into the running production process without manual efforts and changes \cite[p. 146]{Ye20204.0}. This increases the flexibility of manufacturing processes. Digitally connected, the process steps of a production line can be better coordinated, which results in better utilization of available machine capacities, shorter setup-times and less manual intervention. \cite[p. 16]{Acatech2013Recommendations4.0}. 
    \item[] \textbf{Lot Size One} Lot Size One production describes the ability of a production line, to produce each product or item according to the individual buyer's specification. The modularization allows the production of individual products in small quantities at affordable prices by using automated engineering and production capabilities \cite[p. 15]{Acatech2013Recommendations4.0}.
    \item[] \textbf{Service Oriented Business Models} Service oriented business models describe the ability of a company to sell the products usage or performance rather than the product itself, which generates new sources of competitiveness. By using runtime or engineering data to improve maintenance or predict machine fail over, new services can be offered \cite{Bendig2021Equipment-as-a-ServiceIndustry}.
\end{itemize}

However, to realize the horizontal and vertical integration of all components and systems an operational end-to-end communication, also known as interoperability, must be achieved \cite[p. 858]{Uslander2015ReferenceApproach}. This assumes a common language, with consistent vocabulary, semantics and rules, so that independent systems, components as well as humans can communicate with each other. In order to enable a uniform language for the interaction of different stakeholders, numerous standardization efforts by different professional associations such as \ac{ZVEI} and \ac{VDMA} have been undertaken. To focus standardization efforts and ensure a coordinated development, the Plattform Industry 4.0 was founded in 2013. In a first step, a reference architecture model was published in 2016. \ac{RAMI4.0} lays the foundation for a cross-industry discussion about the complex tasks and contexts regarding the implementation of \ac{I4.0} \cite[p. 4]{Heidel2017ReferenzarchitekturmodellIndustrie4.0Komponente}. One of the key concepts of \ac{RAMI4.0} is the \ac{I4.0} component with the \ac{AAS}. The \ac{AAS} is the implementation of the \ac{DT} in the context of \ac{I4.0}. The \ac{AAS} enables a vendor and technology independent virtual representation of physical assets according to defined semantic standards and rules \cite[p. 5]{Wagner2017ThePlant}. The \ac{AAS} acts as an interface between different components and processes and stores all necessary properties and data about a physical asset over its life cycle. The \ac{AAS} makes it possible to integrate value-added services into distributed \ac{CPS} so that new business models can emerge within and across companies. With its cooperation possibilities, the \ac{AAS} will therefore become the basis for all new use cases realized in the fourth industrial revolution \cite[p. 67]{Heidel2017ReferenzarchitekturmodellIndustrie4.0Komponente}.    

Given the fact, that one of the visions of \ac{I4.0} is to make production more flexible and autonomous by enabling decentralized execution of processes, current concepts in the field of engineering and production fall short. To understand this in more detail, a closer look needs to be taken at the current state of the automation pyramid, defined by the \ac{ISA} \cite{InternationalSocietyofAutomation2008EnterpriseIntegration}. Currently, given the production line equipped with some robotic and manual workstations, an engineer needs to obtain requirements from the office floor, in order to design the entire production process and assign the required production resources. The requirements can come hereby in any form, without generally applicable semantic standards, and can be expressed in some supportive high-level languages like GANT- or flowcharts or lists of physical equipment (Bill of Material). The required manufacturing process is then implemented using dedicated \ac{PLC} which execute the corresponding process steps. The results of the individual process steps are reported then to a central higher level supervision and control system for monitoring \cite[p.2, 3]{Wagner2017ThePlant}. The intelligence and know-how to setup and run a production process lies exclusively with the engineer. Each change to the production line and its possible side-effects must be planned in advance. Consequently the production line is highly tailored to a specific product and manufacturing resource and is only flexible to the extent that it was planned in advance \cite[p. 1]{Keddis2015Capability-basedSystems}. In order to realize the advantages of \ac{I4.0}, the allocation of production resources and the definition of the production process can no longer be done manually. This must be realized automatically during operational use \cite[p. 9]{Bock2016Weiterentwicklung4.0-Komponenten}.

To address the shortcomings, companies are moving towards implementing a \ac{SOA}. In a \ac{SOA}, the centralized control of business and manufacturing processes is replaced by a distributed architecture \cite[p. 861]{Uslander2015ReferenceApproach}. This approach is also called capability based engineering and operation of systems \cite[p. 5]{Bayha2020DescribingComponents}. For this purpose, the individual control components and information systems offer their functionalities and capabilities in the form of web services, which communicate with each other via a uniform interface. This is realized by the \ac{DT}, respectively the \ac{AAS} in \ac{I4.0}. A key characteristic of this approach, compared to the current state is, that the services in a \ac{SOA} are loosely coupled. This means, that all services can perform their capabilities independently of each other without being aware of the current production process and product to produce. This leads to a high degree of modularity and resuability and removes possible side-effects when changing the business or manufacturing process \cite[p. 491]{Schicke2020EnablingTwins}. Distributed architectures and especially the \ac{AAS} are thus moving into focus and increasingly take over tasks that were previously performed by centrally located systems. To realize this, capabilities and processes need to be associated with the \ac{AAS} so that they can be executed at the right time in the right way. This way, several temporarily interconnected components can perform more complex operations together as part of a machine or plant network. However, according to the current state of research, \ac{RAMI4.0} and the \ac{AAS} offer only limited methods for designing, operating and monitoring the composition of loosely coupled services in order to execute business and production processes. 

A widely used method for designing, operating and controlling processes within a company is \ac{BPM}. \ac{BPM} enables companies to ensure consistent outcomes and take advantage of improvement opportunities regarding their business processes \cite[p. 1]{Dumas2018FundamentalManagement}. The improvements can be of a very different nature, such as cost reduction, efficiency gains or improved quality due to minimization of error rates. To execute the processes at the right time by the right resource, \ac{BPMS} are used. In addition to their executive function, \ac{BPMS} also document the execution of processes which forms the basis for subsequent analyses and improvements of the processes \cite[p. 345]{Dumas2018FundamentalManagement}. The literature already contains approaches on how BPM can be applied in \ac{I4.0} scenarios. \citet[p. 1444]{Schonig2020IoTExecution} show that by integrating machine and sensor data in a process model, a more comprehensive view on a business and manufacturing processes can be achieved as well as cost reductions and efficiency gains can be realized. For example, when producing raw material on multiple machines, quality problems can be identified more quickly due to the available real-time data and the process can be adjusted  automatically without losing time. Thereby the authors point out, that one of the central requirement during the integration of sensor and machine data into a \ac{BPM} model is the provision of context-specific knowledge describing the machine and sensor data \cite[p. 1447]{Schonig2020IoTExecution}. \citet[p. 2]{Jaenisch2017TheChallenges} point out benefits by using \ac{BPM} to orchestrate processes in a \ac{CPS}, in regards to the the planning, execution and monitoring of distributed production processes. For example, sensor and machine data can be used to detect discrepancies between a predefined process model and the actual process. Also the planning of production processes in distributed systems can be improved, as the individual services in the production process know the goals and procedures in terms of time and cost of other services in the system and can act accordingly. However, they also point out that \ac{BPM} and \ac{CPS} can only mutually benefit from each other if the raw sensor and machine data are enriched with a semantic model so that they become interpretable.    

The increasing decentralization and the move towards a \ac{SOA} shows that there is a particular need in the context of how future business and production processes are planned, executed and optimized in a \ac{I4.0} system. This applies in particular to the orchestration of the components and services in a distributed \ac{CPS}. According to the literature \ac{BPM} could play a role in the planning, execution and orchestration of the services. This assumption requires, that the services are reachable by a standardized interface and thus the sensor and machine data are interpretable by means of a semantic model. \ac{RAMI4.0} and the \ac{AAS} enable the semantic description of sensor and machine data by offering a standardized interface. The \ac{AAS} component could thus become the central control element between the business and manufacturing processes to be executed and the services that expose the functionalities and capabilities of an asset. However, currently the concepts of the \ac{AAS} and \ac{BPM} are defined at different abstraction levels and there is so far no explicit link to bridge the gap. Further research is needed to determine whether BPM can be used to plan, execute and orchestrate services in a decentralized \ac{CPS} with the \ac{AAS} as the central connection and control element between the process steps and the resources required for them. 

\section{Research Question} \label{sec:research-question}
This thesis tries to work out an approach how the planing, execution and orchestration of decentralized services with the help of \ac{BPM} and \ac{AAS} in value networks can be implemented based on the \ac{RAMI4.0} architecture.

To this end, the following questions are to be answered in the course of this thesis:

\begin{itemize}
    \item [] \textbf{Question 1:} \textit{How can \ac{BPM} be classified in the six layers of \ac{RAMI4.0}?}
    
    The first question is intended to provide information about which general requirements \ac{RAMI4.0} defines for the six layers of the architecture axis and to what extend \ac{BPM} can be found in them. To this end, the available information in the individual layers and the dependencies between them will be examined in more detail. The main focus will be on determining the relevant information flow between the layers. The result will clarify the functional requirements defined by \ac{RAMI4.0} and how they can be taken into account in \ac{BPM}.
    
    \item [] \textbf{Question 2:} \textit{How can a link between a process and the required services involved in it be established with the help of the \ac{AAS}?}
    
    The second question builds on the results of the first question and takes an even closer look at the \ac{I4.0} component and the \ac{AAS} contained therein. For this purpose, the structure, contents and characteristic features of the \ac{AAS} are elaborated. Furthermore, a closer look at the interaction model between two \ac{AAS} is taken. In order to determine the link between process and service, the \ac{I4.0} component is mapped in the six layers of \ac{RAMI4.0}. The result of the question will shed light on when and how a service can be an active process participant.
    
    \item[] \textbf{Question 3:} \textit{How can a concrete integration of \ac{BPM} in the \ac{AAS} be realized based on the six layers of \ac{RAMI4.0}?}
    
    The third question summarizes the findings from the previous questions and attempts to make a proposal for the integration of \ac{BPM} into the \ac{AAS} . For this purpose, the functional requirements defined in question 1 and the features and characteristics of the \ac{AAS} identified in question 2 are transferred into a concrete application model. The application model is based on the six layers of \ac{RAMI4.0}. 

\end{itemize}

\section{Structure}
The approach and further chapters of the thesis are as follows:
\begin{itemize}
    \item[] In chapter two, a literature review is carried out, to explain the basic concepts of \ac{I4.0} in greater detail. First, value networks with reference to \ac{I4.0} are explained and the resulting need for a reference architecture model is derived, which is presented afterwards. This is followed by an introduction to the general concepts of a \ac{DT} and the implementation of the \ac{DT} in the \ac{I4.0} component.
    
    \item[] The third chapter takes a closer look at the \ac{AAS}. For this reason, the functional requirements of an \ac{AAS} in \ac{RAMI4.0} are examined and elaborated in greater detail. In particular, the different forms of the \ac{AAS} in value networks will be examined on the basis of defined use cases. Finally, the \ac{AAS} will be mapped into the six layers of \ac{RAMI4.0}.
    
    \item[] In the fourth chapter, the application model based on the obtained results from the previous chapters is presented. For this purpose, a technology-neutral architecture is proposed for implementation, which is backed up with a 4-step implementation guide. The fifth chapter summarizes the results obtained and provides an outlook on further topics. 
\end{itemize}


\section{Definition of terms}
In order to have a consistent definition of important terms within this thesis, some are defined once here:
\begin{itemize}
    \item[] \textbf{Component} The term component is used to describe devices such as a sensor, actuator as well as machines, industrial controllers or software processes and agents. A smart component is characterized by the fact that it provides relevant information, such as production parameters or services, such as the independent determination of the operating status via a communication interface in a standardized form with other system participants.
    \item[] \textbf{Asset} The term asset is used to describe an object in the physical world that has a certain business value for a company and can be described on the basis of properties and characteristics.
\end{itemize}