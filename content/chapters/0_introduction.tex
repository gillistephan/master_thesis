\chapter{Introduction}
This chapter first gives an overview of the motivation and background of the thesis. Subsequently, the research question is derived and expected results are considered in more detail. Finally, an overview of the following chapters is given.
\section{Motivation}
In recent years, the interest in \ac{I4.0} has increased rapidly, both in science as well as in industry. \ac{I4.0} describes the vision of a global network in which machines, robots, warehousing systems and other players in the value chain can autonomously interact with each other in order to control physical resources and processes. The autonomous interaction of these transforms production facilities into smart factories \cite[p.20]{Acatech2013Recommendations4.0}. Smart factories enable therefore end-to-end engineering by connecting the digital and physical world both, horizontally and vertically along the value chain, but in particular also across company boundaries \cite[p.20]{Acatech2013Recommendations4.0} \cite[p.859]{Uslander2015ReferenceApproach}. The components and systems of smart factories should be able to make autonomous decisions based on provided information, execute functions regarding their capabilities, as well as autonomously influence production, order or logistic processes. For this reason autonomously acting components and systems are seen as the key enabler for \ac{I4.0} and will have far-reaching effects on business and production processes. One term that is often mentioned in this context is \ac{CPS}. A \ac{CPS} describes the networking of the physical world of machines, plants and devices with the virtual world \cite{Lee2008CyberBerkeley}. The terms smart factory and \ac{CPS} are therefore used interchangeably in this thesis.

\ac{CPS} and smart factories will open up new use cases for companies of various industries and sectors. In particular the following should be mentioned:
\begin{itemize}
    \item[] \textbf{Plug and produce} Plug and Produce describes the ability of a production resource to configure itself, based on its environment conditions and integrate itself correctly into the running production process without manual efforts and changes \cite{Schleipen2015Requirements4.0} \cite[p.146]{Ye20204.0}. This increases the flexibility of manufacturing processes: digitally connected, the process steps of a production line can be better coordinated, which results in better utilization of available machine capacities, shorter setup-times and less manual intervention. \cite[p.16]{Acatech2013Recommendations4.0}. 
    \item[] \textbf{Lot Size One} Lot Size One production describes the ability of a production line, to produce each product or item according to the individual buyer's specification. The modularization allows the production of individual products in small quantities at affordable prices by using automated engineering and production capabilities \cite[p.15]{Acatech2013Recommendations4.0}
    \item[] \textbf{Service Oriented Business Models} Service oriented business models describe the ability of a company to sell the products usage or performance rather than the product itself, generating new sources of competitiveness. By using run-time or engineering data to improve maintenance or predict machine fail over, new services can be offered \cite{Bendig2021Equipment-as-a-ServiceIndustry} \cite[p.16]{Acatech2013Recommendations4.0}
\end{itemize}

However, to realize the horizontal and vertical integration of all components and systems an operational end-to-end communication, also known as interoperability, must be achieved \cite[p.858]{Uslander2015ReferenceApproach} \cite[p.39]{Acatech2013Recommendations4.0}. This assumes a common language, with consistent vocabulary, semantics and rules, so that independent systems, components as well as humans can communicate with each other. In order to enable a uniform language for the interaction of different stakeholders, numerous standardization efforts by different professional associations such as \ac{ZVEI} and \ac{VDMA} have been undertaken. To focus standardization efforts and ensure a coordinated development, the Plattform Industrie 4.0 was founded in 2013. In a first step, a \ac{RAMI4.0} was published in 2016. \ac{RAMI4.0} lays the foundation for a cross-industry discussion about the complex tasks and contexts regarding the implementation of \ac{I4.0} \cite[p. 858]{Uslander2015ReferenceApproach} \cite[p.4]{Heidel2017ReferenzarchitekturmodellIndustrie4.0Komponente}. One of the key concepts of \ac{RAMI4.0} is the \ac{I4.0} component. The \ac{I4.0} component enables a vendor and technology independent virtual representation of physical assets, also known as digital twin, according to defined semantic standards. \cite[p.67]{Heidel2017ReferenzarchitekturmodellIndustrie4.0Komponente}. The \ac{I4.0} component acts as an interface between different components and systems and stores all necessary properties and data about the physical asset over its life cycle. In the context of \ac{I4.0}, the \ac{I4.0} component is also referred to as \ac{AAS} \cite[p.68]{Heidel2017ReferenzarchitekturmodellIndustrie4.0Komponente}.

Given the fact, that one of the visions of \ac{I4.0} is to make production more flexible and autonomous by enabling decentralized execution of processes, current concepts in the field of engineering and production fall short. To understand this in more detail, a closer look needs to be taken at the current state using a simplified explanation of the automation pyramid, defined by the \ac{ISA}. Currently, given the production line equipped with some robotic and manual workstations, an engineer needs to obtain requirements from the office floor, in order to design the entire production process and assign the required production resources. The requirements can come hereby in any form, without generally applicable semantic standards, and can be expressed in some supportive high-level languages like GANT- or flowcharts or lists of physical equipment (Bill of Material). The required manufacturing process is then implemented using dedicated \ac{PLC} which execute the corresponding process steps. The results of the process steps are reported then to a higher level supervision and control system for monitoring. The intelligence and know-how to setup and run a production line lies exclusively with the engineer. Each change to the production line and its possible side-effects must be planned in advance. Consequently the production line is highly tailored to a specific product and manufacturing resource and is only flexible to the extent that it was planned in advance. \cite[p.1]{Keddis2015Capability-basedSystems}

To address the shortcomings, many companies are moving towards implementing a \ac{SOA}, in which the centralized control of business and manufacturing processes is replaced by a distributed architecture \cite[p.861]{Uslander2015ReferenceApproach} \cite[p.491]{Schicke2020EnablingTwins}. This approach is also called capability based engineering and operation of systems \cite[p.5]{Bayha2020DescribingComponents} \cite[p.2]{Keddis2015Capability-basedSystems}. For this purpose, the individual control components (\ac{PLC}) and information systems offer their functionalities and capabilities in the form of web services, which communicate with each other via a uniform interface: the digital twin. A key characteristic of this approach, compared to the current state, is, that the services are loosely coupled. This means, that all services can perform their capabilities independently of each other without being aware of the current production process and product to produce. This leads to a high degree of modularity and resuability and removes possible side-effects when changing the business or manufacturing process \cite[p.491]{Schicke2020EnablingTwins}. Distributed architectures and digital twins are thus moving into focus and increasingly take over tasks that were previously performed by centrally located systems. To realize this, capabilities and processes need to be associated with the digital twin so that they can be executed at the right time in the right way. This way, several temporarily interconnected components can perform more complex operations together as part of a machine or plant network. However, according to the current state of research, \ac{RAMI4.0} and the \ac{AAS} offer only limited methods for designing, operating and monitoring the composition of loosely coupled services in order to execute business and production processes. 

A widely used method for designing, operating and controlling processes within a company is \ac{BPM}. \ac{BPM} enables companies to ensure consistent outcomes and take advantage of improvement opportunities regarding their business processes \cite[p. 1]{Dumas2018FundamentalManagement}. The improvements can be of a very different nature, such as cost reduction, efficiency gains or improved quality due to minimization of error rates. To execute the processes at the right time by the right resource, \ac{BPMS} are used. In addition to their executive function, \ac{BPMS} also document the execution of processes which forms the basis for subsequent analyses and improvements \cite[p. 345]{Dumas2018FundamentalManagement}. The literature already contains approaches on how BPM can be applied in \ac{I4.0} scenarios. \citet[p. 1444]{Schonig2020IoTExecution} show that by integrating machine and sensor data in a process model, a more comprehensive view on a business process can be achieved as well as cost reductions and efficiency gains can be realized. For example, when producing raw material on multiple machines, quality problems can be identified more quickly due to the available real-time data and the process can be adjusted without losing time. Thereby the authors point out, that one of the central requirement during the integration of sensor and machine data into a BPM model is the provision of context-specific knowledge describing the machine and sensor data \cite[p. 1447]{Schonig2020IoTExecution}. \citet[p. 2]{Jaenisch2017TheChallenges} point out benefits by using \ac{BPM} to orchestrate processes in a \ac{CPS}, in regards to the the planning, execution and monitoring of distributed production processes. For example, sensor and machine data can be used to detect discrepancies between a predefined process model and the actual process. Also the planning of production processes in distributed systems can be improved, as the individual services in the production process know the goals and procedures in terms of time and cost of other services in the system and can act accordingly. However, they also point out that \ac{BPM} and \ac{CPS} can only mutually benefit from each other if the raw sensor and machine data are enriched with a semantic model so that they become interpretable.    

The increasing decentralization of production processes shows that there is a particular need in the context of orchestrating the involved components and services. According to the literature \ac{BPM} could play a role in the orchestration of services assuming that these services are reachable by a standardized interface and the sensor and machine data are interpretable by means of a semantic model. The \ac{AAS} represents such a semantic model for describing sensor and machine data as well as offering a standardized interface. However, currently the \ac{AAS} and \ac{BPM} are defined at different abstraction levels and there is so far no explicit link to bridge the gap. While the level of activities within the process and its execution are in the foreground at \ac{BPM}, the \ac{AAS} is about the lower level input regarding the retrieval of information that relate to these process activities. Further research is needed to determine whether BPM can be used to orchestrate services and how the \ac{AAS} can be used to provide machine-interpretable data. 

\section{Research Question}
This thesis tries to work out an approach how the orchestration of decentralized services with the help of \ac{BPM} and \ac{AAS} can be implemented.

To this end, the following questions are to be answered in the course of this thesis:

\begin{itemize}
    \item [] \textbf{Question 1:} \textit{How can \ac{BPM} be classified in the six layers of \ac{RAMI4.0}?}
    
    The first question is intended to provide information about which general requirements \ac{RAMI4.0} defines for the six layers of the architecture axis and to what extend \ac{BPM} can fulfill this. To this end, the available information in the individual layers and the dependencies between them will be examined in more detail. The main focus will be on determining the relevant information flow. 
    
    \item [] \textbf{Question 2:} \textit{}

\end{itemize}

\section{Structure}
The approach of the thesis as follows:
\begin{itemize}
    \item In a first step, a literature review is carried out. For this purpose, the basic concepts of \ac{I4.0}, such as \ac{RAMI4.0} and \ac{AAS} are presented in greater detail. 
\end{itemize}






