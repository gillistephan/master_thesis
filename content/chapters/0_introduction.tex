\chapter{Introduction}

\section{Motivation}
In recent years, the interest in \ac{I4.0} has increased rapidly, both in science as well as in industry. I4.0 describes the vision of moving away from simple exchange of data between control components to a cooperation based approach, in which control components can autonomously interact with each other in order to control physical processes. That is supposed to be both horizontally and vertically along the value chain, but in particular also across company boundaries \cite{Acatech2013} \cite{}. Control components can be any building block of a technical system like a sensor, machine, production-cell or even a plant. Control components should be able to make automated decisions based on provided information, execute functions regarding their capabilities, as well as autonomously influence production-, order- or logistic processes, by interacting with other control components. Therefore these autonomously acting control components are seen as the key enabler for industry 4.0 and will have far-reaching effects on business, manufacturing and production processes of companies. When connected to each other in a dynamic, self-organized network without any hierarchical control instance, these control components are referred to as \ac{CPS} \cite{Lee2008}.

\ac{I4.0} and \ac{CPS} will open up new use cases for companies of various industries and sectors. In particular the following should be mentioned:
\begin{itemize}
    \item Plug and produce describes the ability of a control component to configure itself, based on its environment and environment conditions and integrate itself correctly into the running production process without manual efforts and changes \cite{Schleipen2015} \cite{Ye2020}
    \item Lot Size One Production describes the ability of a production line, to produce each product or item according to the individual buyer's specification
    \item Service oriented business models describe the ability of a company to sell the products usage or performance rather than the product itself \cite{Bendig2021}
\end{itemize}

This results in several advantages for companies like:
\begin{itemize}
    \item Increased flexibility: digitally connected, the process steps of a production can be better coordinated, which results in better utilization of available machine capacities, shorter setup-times of production lines and less manual intervention
    \item Customized Mass Production: the modularization of production allows individual products in small quantities at affordable prices by using automated engineering capabilities
    \item Generation of new sources of competitiveness: using run-time or engineering data to improve maintenance or predict machine fail over
\end{itemize}

However, to realize the autonomous changeability of processes and enable automated decision making, an operational end-to-end communication between all production assets including sensors and \ac{PLC}, servers and programs in the IT like \ac{ERP} must be implemented. This assumes a common language, with consistent vocabulary, semantics and rules, so that independent systems, components as well as humans can communicate with each other.

In order to enable a uniform language for the interaction of different stakeholders, there have been numerous standardization efforts in recent years by different professional associations such as \ac{ZVEI} and \ac{VDMA}. To focus standardization efforts and ensure a coordinated development, the Plattform Industrie 4.0 was founded 2013 under the umbrella of the Ministry of Economy and Energy. In a first step, a \ac{RAMI4.0} was published in 2016, to enable a cross-industry basis for discussion about the complex tasks and contexts regarding the implementation of \ac{I4.0}. One of the key concepts of \ac{RAMI4.0} is the I4.0 component, which enables a technology- and platform independent virtual representation (digital twins) according to semantic standards of physical assets \cite{Heidel2017}. In the context of I4.0, the I4.0 component is referred to as \ac{AAS}. The \ac{AAS} acts as an interface between different stakeholders and stores all necessary properties and data about the physical asset.

Given the fact, that one of the visions of \ac{I4.0} is to make production more flexible, efficient and autonomous, current concepts in the field of engineering and production fall short. To understand this in more detail, a closer look needs to be taken at the current state using a simplified description. At the moment, given the production line equipped with some robotic and manual workstations as well as transportation systems, an engineer needs to obtain requirements from the office floor, in order to design the entire production process and assign the required production resources. The requirements can come hereby in any form and can be expressed in some supportive high-level languages like Gant- or flowcharts or lists of physical equipment like Bill of Material (BOM). The required process is then implemented using dedicated \ac{PLC} and robot programming. The intelligence and know-how to setup and run a production lies exclusively with the engineer. Each change to the production line and its possible side-effects must therefore be planned in advance. This means, that the resulting systems are only flexible to the extent that they were planned in advance.
