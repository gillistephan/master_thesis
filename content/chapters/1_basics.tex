\chapter{Basics}
This chapter describes the concepts necessary for a basic understanding of the research question. To this end, the first step is to introduce the concept of value networks. Specifically, the implementation of these in the context of \ac{I4.0} (\ac{RAMI4.0}) is discussed using practical examples. This is intended to demonstrate the need for a reference model for \ac{I4.0}, which will be explained in more detail in the next step. Since the \ac{I4.0} component plays a central role within \ac{RAMI4.0}, it will be discussed afterwards. The \ac{I4.0} component corresponds to the digital twin in the industrial context. For this reason, a general introduction to digital twins is given first and then the reference to the \ac{I4.0} component is established. In a final step, \ac{BPM} is presented with the necessary elements for modeling.

\section{Value Networks in Industrie 4.0} \label{sec:valuenetworks}
\citet[p. 6]{Sturgeon2001HowNetworks} defines value chains as the connection of value-adding activities of a company in order to create a product or a service. The activities can have complex relationships to each other and can take place at different locations. Within a company, all departments that are involved in the production or sale of a product are part of the value chain \cite[p. 10]{Sturgeon2001HowNetworks}. \citeauthor{Sturgeon2001HowNetworks} refers to this type of networking as vertical networking. Whereas the value chain focuses on the individual product, the term value network covers all value chains in which a company operates. A value network can include various cooperating as well as legally independent companies. To this end, each company within the network contributes its core competencies in order to achieve collaborative competitive advantages and create value through their mutual interaction \cite[p. 3]{Bach2010GeschaftsmodelleGrundlagen}. Consequently, value networks have in addition to the vertical networking also a horizontal networking, namely the connection across company boundaries. Value networks have therefore the potential to create new product-service bundles both, within and across companies traditional scope. Thus, companies can use value networks on the one hand to grow their existing core business, but also expand their network by tapping into new market segments \cite[p. 22]{Acatech2013Recommendations4.0}.

In the context of \ac{I4.0} especially the horizontal integration through value networks has increased. The horizontal integration is driven by the vision of autonomously acting resources within a network that can configure themselves, automatically respond to different situations, and make decisions accordingly. Horizontal networking in \ac{I4.0} is therefore almost exclusively technology-based \cite[p. 20, 21]{Acatech2013Recommendations4.0}. The control of the resources is made possible by the fact, that all relevant information is available in real-time through the networking of all instances of value creation. This enables constant evaluation and optimization of processes on the basis of the collected data \cite[p. 24]{Acatech2013Recommendations4.0}. 

The horizontal connection of partners within a value network in \ac{I4.0} is typically realized via \ac{IIOT} platforms or digital ecosystems. An \ac{IIOT} platform is characterized by the interaction of entities, information systems, business partners and people to drive production and operations in which each of the partners makes its specific added value available to other partners in the network \cite[p. 5]{Falk2020ValueChina} \cite[p. 2]{Falk2021DigitaleEntwicklung}. In the area of \ac{IIOT} platforms, two main types are emerging. Platforms, with a generic approach, that aim to collect data in general about the use and operation of spatially distributed assets in order to gain insights \cite[p. 11]{Falk2021DigitaleEntwicklung}. On the other hand, the emergence of \ac{IIOT} platforms and ecosystems around specialized offerings of companies can be seen \cite[p. 13]{Falk2021DigitaleEntwicklung}. While the former platforms predominantly take an open platform approach, the latter are mostly closed platforms. The degree of openness of a platform refers to the possibility of integrating further partners and making the platforms content accessible to them. 

A concrete use case within an \ac{IIOT} platform represents predictive maintenance. Predictive maintenance by definition refers to a maintenance process based on the evaluation of process and machine data collected by sensors \cite[p. 1670]{Selcuk2016PredictiveTrends:}. The collected data is either sent in its raw or aggregated form to the cloud, where it is stored and made available for different partners in the network. The real-time processing of the underlying data makes forecasts possible that form then the basis for needs-based maintenance. Thus, it can improve the economic efficiency of a real world asset, predict the optimal time for maintenance and improve the performance by minimizing unplanned downtime. While costs are not the only benefit of implementing predictive maintenance several other motives underlie the considerations to replace preventive maintenance by predictive maintenance. For example during agreed maintenance intervals for hardware assets, most often hardware parts are replaced, although they were neither defective nor threatened to fail \cite[p. 1672]{Selcuk2016PredictiveTrends:}. A concrete example of an implementation is provided by \citeauthor{Cavalieri2020AShell}. While the collection of the machine data is carried out by the operator itself, a service provider takes over the implementation of the prediction and decision model in the cloud \cite[p. 11]{Cavalieri2020AShell}. Seamless integration is enabled by interoperable concepts for data collection, processing as well as decision making. This allows vendor-independent functionalities for predictive maintenance to be developed by one vendor, which can be use by multiple partners in the network, which increases flexibility and efficiency \cite[p. 9]{Cavalieri2020AShell}.

As outlined in the case of predictive maintenance, value networks are the basis for manufacturing companies to generate new revenue streams and gain a competitive advantage, by connecting, merging and evaluating data on the use and operation of the entire product life cycle. While this example is of simple nature, value networks are often much more complex.

\begin{figure}[h]
\includegraphics[scale=0.35]{content/pictures/digital_ecosystems_bcg.png}
\caption{Digital and Data Ecosystems}
\source{\cite{Russo2018HowCompetition}}
\label{fig:valuenetworksi40}
\end{figure}

Figure \ref{fig:valuenetworksi40} provides an overview of a value network in \ac{I4.0} with their roles and actors. Typically, four roles can be seen in \ac{I4.0} value networks. Usually, they are identical with the players seen along the traditional value chain. These are the platform operator, the OEMs, their customers as well as suppliers. In addition to the traditional suppliers, new service providers from the digital ecosystem play an important role. Each partner contributes a small or larger part to the network by bringing their unique capabilities for a special area. For example, the domain expert provides the necessary expertise to interpret the specific data collected by sensors and machines, while the data broker ensures, that the data is accessible, accurate and consistent. Based on the provided information of the domain expert, the analytic provider implements algorithms using machine learning or artificial intelligence to provide insights on a massive scale \cite{Russo2018HowCompetition}. Practical examples of value creation networks in \ac{I4.0} are the Airbus Skywise Platform or the Volkswagen Industrial Cloud. Skywise was launched 2017 as an internal project at Airbus with the aim of analyzing millions of data points collected during the operation of an aircraft, to get a close-up view of its condition in order to avoid defects and downtime \cite{Hanke2019AirbusWerden}. Airbus has since expanded it's ambition: Skywise aims to be the leading platform for the aircraft industry to revolutionize the way aircraft are designed, built and operated. To this end, Airbus has opened their platform to suppliers and service providers. 

This type of development is typical: First, the collected data is used to improve the companies own products, services and the internal processes (vertical integration). In a second step an open approach is taken to provide new, data-driven products and service offerings in collaboration with network partners (horizontal integration). This can be seen in most manufacturing companies. They are succeeding in the first steps, namely the formation of digital units and, in some cases, the collection and evaluation of data for the realization of cost savings and quality improvements. While these efforts by companies are mainly focused on very simple and specific use cases, a more generic solution is needed to generate new revenue models. But this is where most companies struggle, because they lack the knowledge or capital needed to build the capabilities \cite[p. 19]{Arnold2018DigitaleMittelstand}.

However, in order to reach the point of developing new business models, many companies need to enter into cooperative ventures in order to take advantage of each others capabilities \cite{Russo2018HowCompetition}. For example, due to the lack of expertise in centralizing and analyzing large amounts of data using artificial intelligence and machine learning, Airbus entered into a cooperation with the American company Palantir \cite{Hanke2019AirbusWerden}. Another example is Volkswagen's announced cooperation with Amazon Webservices regarding the Industrial Cloud \cite{VolkswagenAG2019WirLive}. While Volkswagen contributes its expertise from the areas of vehicle construction and production, Amazon Webservices provides its knowledge from the areas of cloud, logistics and platform operation. With the announcement of the development of the Industrial Cloud, Volkswagen is pursuing the goal of bringing together all data from machines and systems from all factories across the entire group. Volkswagen has designed the Industrial cloud as an open industry platform so that almost the entire global supply chain, including all suppliers and partner companies, can be integrated. Likewise, Volkswagen has announced that it is considering opening the platform to other vehicle manufacturers.

The open approach of the two companies Airbus and Volkswagen opens up new use cases for cooperation and value creation: By participating on the industrial cloud of Volkswagen, all suppliers and component manufacturers obtain more and better usage information about their products and can thus bring their product performance to the next level. This holds true especially in many cases since suppliers and component manufacturers do not have the necessary financial means to build it up themselves \cite{Russo2018HowCompetition}. The same applies to Airbus. However, this approach does not only have advantages for increasing the performance of the suppliers own products, it also enables easier access for them to services and products from other suppliers of the Volkswagen and Airbus network. 

As outlined using the example of Airbus and Volkswagen: through \ac{I4.0}, value creation networks with different companies emerge, which are characterized by complex mutual relationships. To make the complexities understandable and manageable for all partners in the value network, a common technical standard is needed (as demonstrated by the simple example of predictive maintenance). Processes and interfaces are required that define collaboration and the information to be exchanged. This can be achieved by a reference model. A reference model describes the technical description and implementation of these agreements. It thus provides a general pattern for the products and services of all cooperating companies and forms the framework for both the structuring and the development, integration and operation of the relevant technical systems. A value network consists of a wide variety of companies with different business models. The reference model has the task of bringing together the different perspectives into a common and unified view. For this purpose, it is necessary to agree on basic structuring principles as well as interfaces and data.

\section{Referenzarchitekturmodell 4.0} \label{sec:rami40}
\ac{RAMI4.0} is a reference model, which enables the interdisciplinary technical data description of an asset over the entire life cycle from design to disposal as well as the consistent representation of the complex relationships of partners within a value network \cite[p. 4]{Heidel2017ReferenzarchitekturmodellIndustrie4.0Komponente}. \citet[p. 23]{Arnold2018DigitaleMittelstand} emphasize several advantages of \ac{RAMI4.0} when applied by all partners in the network. Applying \ac{RAMI4.0}, collaborations between network partners can be realized more easily and quickly, which saves transaction costs and increases efficiency. This is in particularly true because the interfaces and processes required for collaboration are standardized. A good example of how transaction costs can be reduced by applying \ac{RAMI4.0} is given by \citet[p. 11]{Bayha2020DescribingComponents}. If the interaction between two \ac{I4.0} compliant components is to be established, the transaction costs of the implementation depend on the degree of formalization of these components. The most informal way of describing interfaces and processes is when the two developers of the components use natural language to exchange the meaning of the messages and agree on a common exchange format. The authors note that this causes a high coordination effort on both sides and ultimately results in higher transaction costs for the implementation. This is confirmed empirically by \citet[p. 14]{Arnold2018DigitaleMittelstand}, which show that horizontal networking in companies is slowed down primarily by the high transaction costs caused by non-uniform standards. To reduce the coordination effort, standardized information models like \ac{OPCUA}\footnote{\ac{OPCUA} is a machine-to-machine protocol for industrial automation, which is not further discussed. More information can be found on https://opcfoundation.org/about/opc-technologies/opc-ua/} or eclass\footnote{eclass is a standard for master data management and semantics for digitization which is not further discussed. More information can be found on https://www.eclass.eu/}, as proposed by \ac{RAMI4.0}, can be used to define a uniform vocabulary and structure. A good graphical representation of the effect can be found on page 12 in \citet{Bayha2020DescribingComponents}. In addition to reducing transaction costs, \ac{RAMI4.0} can help companies develop new business ideas and models as it provides a comprehensive overview of processes, customer requirements and competitors. Furthermore \ac{RAMI4.0} can be used to develop new software solutions in conformity with the \ac{I4.0} concept and adapt existing solutions accordingly. The latter two advantages are not discussed further, but can be read in greater detail in \citet[p. 24 ff]{Arnold2018DigitaleMittelstand}.

The main object of consideration in \ac{RAMI4.0} is an asset. An asset is described as an object that has a certain business value for a company. The term therefore not only applies to physical asset but also includes ideas and software in its definition \cite[p. 31]{Heidel2017ReferenzarchitekturmodellIndustrie4.0Komponente}. \ac{RAMI4.0} is a complex model due to the variety of processes, products and services that can be mapped in it. To simplify the complex aspects and interrelationships of an asset in the context of \ac{I4.0}, \ac{RAMI4.0} consists of three dimensions: Layers, Life Cycle \& Value Stream and Hierarchy Levels. The individual dimensions are divided into sub-models to ensure that a common understanding for discussion emerges. The subdivision into sub-models makes it possible to present the relevant aspects of an asset at each point in the life cycle. Figure \ref{fig:rami40} shows the three axis of the model which are described in detail below.

\begin{figure}[h]
\includegraphics[scale=0.5]{content/pictures/rami_4.0_zvei.jpg}
\caption{Referenzarchitekturmodell Industry 4.0}
\source{\cite{ZVEI2015TheZvei.org}}
\label{fig:rami40}
\end{figure}


\textbf{Life Cycle \& Value Stream} As a result of customized products, companies are concentrating more on their core competencies and increasingly outsource value-added activities \cite[p. 11]{Arnold2018DigitaleMittelstand}. This leads to a stronger linkage and dependency of the partners in the value chain as described in \ref{sec:valuenetworks}. On the Life Cycle \& Value Stream axis, these relationships of the partners in the value chain as well as the life cycle of an asset can be mapped. The life cycle of an asset is divided into type and instance \cite[p. 43]{Heidel2017ReferenzarchitekturmodellIndustrie4.0Komponente}. In the first phase of the asset, which includes the planning, development and construction of the asset, the asset type is created. The asset type acts as a blueprint from which an instance can be create in a second phase. The second phase describes the production of the previously developed asset type. From an asset type several instances can be created, which are identified by means of a unique ID (e.g. serial number). The information created during production, such as material used, is added then to the instance. After production, the asset is used by the customer, resulting in usage data, which is also added to the instance \cite[p. 43]{Heidel2017ReferenzarchitekturmodellIndustrie4.0Komponente}. The recording of usage data by the customer enables the manufacturer of the asset to use it to gain insights for improving the product, by incorporating it back into the asset type.
    
\textbf{Layers} The vertical axis of the architecture model describes the structural properties of an asset or their combination in six layers. Since the theoretical foundation of the layers is very abstract, an example is added to each layer, where needed, to clarify the intention. The example is based on the use case of lot size one production, in which the customer can order an individually configured product which is delivered within the same day \footnote{The use case description is based on the example of \citet[p. 39 - 41]{Anderl2016Softwarearchitekturen4.0}. The focus of the authors' mapping is on software architecture and was not felt to be relevant as a whole for this thesis. Relevant parts are therefore slightly modified and irrelevant parts omitted.}. 
    
The lowest layer is the \textit{Asset-Layer}. It represents the real asset in the physical world. \cite[p. 46]{Heidel2017ReferenzarchitekturmodellIndustrie4.0Komponente}. In the present use case, this includes the product to be customized, the production plant including all machines and means of transport. 
    
The \textit{Integration-Layer} connects the physical world with the virtual world by transferring generated events of the physical world (for example, the change of the state of a machine) into the virtual world \cite[p. 46]{Heidel2017ReferenzarchitekturmodellIndustrie4.0Komponente}. For the use case of the individual production, this could be the transmission of an event that a needed resource for the production process configured itself according to the product specification and is ready for use.  
    
The \textit{Communication-Layer} describes the access to the asset's information. This includes the protocol to use, the data format and the transmission of the data. The ensure interoperability, the protocol and data format must be according to \ac{I4.0} standards such as \ac{OPCUA}. This allows assets from different layers and networks to connect to each other \cite[p. 47]{Heidel2017ReferenzarchitekturmodellIndustrie4.0Komponente}. Since this layer is only responsible for the technical transmission of \ac{I4.0} compliant messages, an example is omitted. For a list of suitable protocols and standards for communication, please refer to \citet[p. 49]{Heidel2017ReferenzarchitekturmodellIndustrie4.0Komponente}.  
    
The provision of the structured data, the processing of the events as well as the analysis of these takes place in the \textit{Information-Layer} \cite[p. 51]{Heidel2017ReferenzarchitekturmodellIndustrie4.0Komponente}. The information layer thus provides an important basis for further applications such as predictive maintenance, which are described in the functional layer. Events occurring during the production process of the individual product, such as that the resource is ready for use or the current state of production are stored and evaluated in the communication layer and made available for further processing or display in the functional layer. Based on the current production progress, the expected delivery time could be calculated and compared with the promised delivery time.
    
Services, capabilities and functions provided by the asset are mapped in the \textit{Functional-Layer} \cite[p. 51]{Heidel2017ReferenzarchitekturmodellIndustrie4.0Komponente}. Within the function layer, a distinction is made between basic, process and manufacturer-specific functions. Basic functions are manufacturer-independent functions that base their functionality on \ac{I4.0} compliant data from the information layer. One example for this is predictive maintenance as shown in \ref{sec:valuenetworks}. Process functions on the other hand are production specific functions such as forming, cutting or drilling. Manufacturer-specific functions are those that are not \ac{I4.0} compliant and are proprietary such as the application of a patented process. In the case of customized production, the business model would be mapped to the production process here. This would be done by defining the parts of the product, the customer is able to individualize and which not. The defined elements would then be integrated into the process functions, via standardized interfaces (Bill of Material in the digital twin), as well as the ordering interface defined by the business layer.  
    
The \textit{Business-Layer} describes the business model and is responsible for orchestrating the services defined in the functional layer. It thus describes the business models and the resulting overall processes \cite[p. 53]{Heidel2017ReferenzarchitekturmodellIndustrie4.0Komponente}. In the present use case, the focus would be on the manufacturing and sale of individual products created by the customer, as well as the flexible production system. Furthermore the tools (e.g. apps or webshop) used by the customer to individualize his products would also be located here.
    
\textbf{Hierarchy Levels} This axis refers to the functional classification within an \ac{I4.0} factory and ranges from product to technical plant to the networked world. This axis contains the previous elements of the automation pyramid and extends them by three levels: Product, field devices and the connected world \cite[p. 44]{Heidel2017ReferenzarchitekturmodellIndustrie4.0Komponente}. By connected world, connections to other factories, locations or partners outside of the factory are meant. The product layer was included because products can influence their production autonomously while being manufactured, thus becoming smart products in \ac{I4.0} \cite[p. 21]{Acatech2013Recommendations4.0}. Field devices are understood to be intelligent sensors and actuators that provide the link between the physical and digital world.  

\section{Digital Twin}





