\chapter{Summary} \label{chap:summary}

\section{Conclusion}
This thesis deals with the \ac{DT} in the context of \ac{I4.0}. Specifically, it examines the role of the \ac{DT} in value networks and how business and production processes can be mapped in it, so that it becomes the central control element in \ac{CPS}. To this end, a first chapter provides an introduction to value networks and their realization in the context of \ac{I4.0}. It shows, that the realization of value networks in \ac{I4.0} is predominantly technology-based and takes place on platforms. The realization of value networks via platforms in \ac{I4.0} opens up new opportunities for companies. However, the possibilities that can be realized via \ac{I4.0} platforms often go hand in hand with complex relationships between the partners. This is on the one hand because the technology-based approach introduces new processes and interfaces that are needed to exchange data between the partners. On the other hand, there are new participants in value creation, especially from the digital ecosystem, offering specialized services. To make the complexities manageable, a reference architecture model, namely \ac{RAMI4.0}, is presented. \ac{RAMI4.0} makes it possible to map the complex relationships between the partners of a value network and enables the interdisciplinary technical data description of an asset over the entire life cycle. With this in mind, \ac{RAMI4.0} introduces three dimensions to simplify the description of assets. These are layers, life cycle and value streams as well as hierarchy levels. The information about an asset is stored in the layers, the asset's life cycle is mapped in the life cycle and value stream, and the connection between the asset and the connected world is established in the hierarchy levels. A central element for the data-based technical description of assets in \ac{RAMI4.0} is the \ac{I4.0} component with its \ac{AAS}. In the context of \ac{I4.0}, the \ac{AAS} corresponds to the digital twin and allows a technology-independent description of assets. For this purpose, a general introduction to \ac{DT} in the context of \ac{CPS} is given and the connection to the \ac{I4.0} component and \ac{AAS} is established.  In a second chapter, a closer look at the \ac{AAS} and its characteristics is taken. The \ac{AAS} is a technology-independent description of assets with the help of standardized submodels. The characteristic features of an \ac{AAS} are illustrated on the basis of three use cases: the digital nameplate, predictive maintenance as well as plug and produce. Next to the description of the properties in submodels, three important features of the \ac{AAS} are identified: These are the role of the \ac{AAS} in value networks, the distribution and their composition. The communication between the individual components in an \ac{I4.0} system is realized with the help of the \ac{I4.0} language, which is manifested in the submodels of the \ac{AAS}. The communication between the individual \ac{AAS} thus enables components in a system to act autonomously by responding to requests from other components and making decision based on this. This makes the \ac{AAS} the central control element in \ac{I4.0} systems. In the last chapter of the thesis, the gained knowledge is transferred into a technology independent architecture concept. For the implementation of the concept, a step-by-step guide is introduced, which establishes the connection between the \ac{AAS} and \ac{BPM}. Finally, a roadmap is introduced that can be used to derive the development path of companies towards \ac{RAMI4.0} and the introduction of the \ac{AAS}.

In the course of the thesis, the questions raised in \ref{sec:research-question} can be answered as follows:

\begin{itemize}
    \item [] \textbf{Question 1:} \textit{How can \ac{BPM} be classified in the six layers of \ac{RAMI4.0}?}
    
    To answer the question, the layers in \ac{RAMI4.0} are examined in more detail with regard to their information stored and the flow of information between the individual layers. It is shown that the information in the layers of \ac{RAMI4.0} can be exchange either within the layer itself or between neighboring layers. The information layer is particularly worthy to mention. The information layer is a central layer in the reference model, since it stores and processes the data transmitted from the communication layer in an interoperable manner. This lays the foundation for the implementation of services in the functional layer and the mapping of business models in the business layer. Contrary to the definition of \ac{RAMI4.0}, it can thus be stated that \ac{BPM} includes three layers: the information, the functional and the business layer. These are mutually dependent. To implement business and production processes according to \ac{RAMI4.0}, the information, functional and business layers must be provided. 
    
    \item [] \textbf{Question 2:} \textit{How can a link between a process and the required services involved in it be established with the help of the \ac{AAS}?}
    
    In order to answer the question, the structure, contents and characteristic features of the \ac{AAS} are elaborated. Likewise, a look is taken at the interaction of \ac{I4.0} components that describe a business model through a process. It is shown that the mapping of functionalities and properties of a physical asset is realized with the help of standardized submodels in the \ac{AAS}. A submodel represents exactly one general or specific aspect of an asset. In order to establish the relationship between a process and the services involved with the help of the \ac{AAS}, the \ac{AAS} must take at least and active role in value networks. This means that the \ac{AAS} has a representation in the functional layer of \ac{RAMI4.0} and provides its functionalities in the form of services. Its active role within value networks allows the \ac{AAS} to communicate and make decisions based on the interaction with other components in the network. Passive \ac{AAS} cannot be active participants in a process because they have no representation in the functional layer.
    
    \item[] \textbf{Question 3:} \textit{How can a concrete integration of \ac{BPM} in the \ac{AAS} be realized based on the six layers of \ac{RAMI4.0}?}
    
    The integration of \ac{BPM} into the \ac{AAS} based on the six layers of \ac{RAMI4.0} can be realized via submodels of the \ac{AAS} by anchoring a process description in the \ac{AAS} using \ac{BPMN}. By doing so, the business models defined in the business layer are translated into processes that are modeled in \ac{BPMN}. The execution of the processes can be performed using workflow engines. To this end, the workflow engine calls the activities defined in the process sequentially. An activity corresponds to exactly one functionality that is defined in a submodel of the \ac{AAS} and exposed via the functional layer of \ac{RAMI4.0}. The actual implementation of the integration takes place in four successive steps: Definition of composition, design of IT-platform, design of information model and modeling of the business and manufacturing process. The composition determines the digitization of the asset: One \ac{AAS} for each component, or one \ac{AAS} that composes multiple components. The IT-platform defines the technology for the implementation of the individual layers, and the design of the information model defines the properties and functionalities of the asset, which is transferred into submodels of the \ac{AAS}.

\end{itemize}

\section{Critical Reflection}

The thesis tries to build a bridge between the technical and business aspects in \ac{RAMI4.0} and its containing \ac{AAS}. During the elaboration, it was found that \ac{RAMI4.0} is a complex reference model that tries to company many technical and business aspects, which makes it difficult to derive a generally valid and understandable architecture that takes all the relevant requirements defined into account. It is a great challenge to prepare and present the essential information of the individual aspects in a holistic way from a business and technical perspective. As a result, the acceptance and implementation of \ac{RAMI4.0} are low so far. To increase acceptance and simplify the implementation of \ac{RAMI4.0} and the \ac{AAS}, a further breakdown of the relevant aspects must be made. To do so, the technical and business aspects of the individual layers of \ac{RAMI4.0} must be made available in more detail and in a simplified form. This thesis can be used as a starting point, as it provides an overview and first approaches to implementation. This requires further specification of the individual steps defined in chapter \ref{sec:design} for implementing the proposed architecture. To do so, the technological and business aspects must be addressed separately. For the technical aspect,  in particular, the information, functional and business layers of \ac{RAMI4.0} are to be considered more closely and the role of the \ac{AAS} hereby. To to this, the technical requirements for the individual layers must be specified in more detail and the defined approaches must be tested independently and their feasibility evaluated in prototypes. For the economic analysis, the business value in particular must be worked out and calculated, which can be realized during the implementation of \ac{RAMI4.0} and the \ac{AAS}. In conclusion, it can be said, that the question of this thesis represents a very extensive subject area, so that the result of the thesis provides more of an overview with initial approaches, rather than a concretely applicable concept. In order to develop a concrete application concept, the question would have had to be limited to a specific field of application.