\chapter{Summary} \label{chap:summary}

This thesis deals with the \ac{DT} in the context of \ac{I4.0}. Specifically, it examines the role of the \ac{DT} in value networks and how business and production processes can be mapped in it, so that it becomes the central control element. To this end, a first chapter provided an introduction to value networks and their realization in the context of \ac{I4.0}. It was shown, that the realization of value networks in \ac{I4.0} is predominantly technology-based and takes place on platforms. The realization of value networks via platforms in \ac{I4.0} opens up new opportunities for companies. However, the possibilities that can be realized via \ac{I4.0} platforms often go hand in hand with complex relationships between the partners. This is on the one hand because the technology-based approach introduces new processes and interfaces that are needed to exchange data between the partners. On the other hand, there are new participants in value creation, especially from the digital ecosystem, offering specialized services. To make the complexities manageable, a reference architecture model, \ac{RAMI4.0}, was presented. \ac{RAMI4.0} makes it possible to map the complex relationships between the partners of a value network and enables the interdisciplinary technical data description of an asset over the entire life cycle. With this in mind, \ac{RAMI4.0} introduces three dimensions to simplify the description of assets 